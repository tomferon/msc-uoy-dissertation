\documentclass[a4paper]{article}

\usepackage{lastpage}
\usepackage{fancyhdr}
\pagestyle{fancy}
\fancyhead{}
\renewcommand{\headrulewidth}{0pt}
\fancyfoot{}
\fancyfoot[LE,RO]{\thepage \thinspace\thinspace of \pageref{LastPage}}
% \fancyfoot[RE,LO]{Dissertation}

\usepackage[latin9]{inputenc}
\usepackage{amsmath}
\usepackage{amssymb}
\usepackage{graphicx}

\begin{document}

FIXME: Quick intro to the topic and explanation of what's to come.

\section{Preliminaries}

% Develop a preliminaries section, in which the tools of stochastic Ito calculus are introduced. Assume that your reader has background in the standard undergraduate modules in mathematics (including analysis, measure theory, probability theory and statistics), but introduce all the notions concerning stochastic processes from scratch. This section should not contain any proofs. Simply set up and state the needed results, providing citations to sources. While developing this section, restrict to the setup needed for the model from [2].

FIXME: Definition of stochatic process

FIXME: Definition of martingale

FIXME: Definition of Wiener process

FIXME: Definition of stochastic integral

FIXME: Definition of It\^o process

FIXME: Definition of stochastic differential equation

FIXME: Explanation of notation as stochastic differentials

FIXME: Uniqueness of characteristics of It\^o processes

FIXME: It\^o formula

FIXME: Multi-dimensional It\^o formula

FIXME: Physical v risk-neutral probability

FIXME: Pricing with risk-neutral probability

FIXME: Girsanov theorem

FIXME: Multi-dimensional Girsanov theorem

FIXME: Martingale representation theorem

FIXME: Multi-dimensional martingale representation theorem

\pagebreak
\section{Multi-asset Black-Scholes model}

The one-dimensional Black-Scholes consist of one risk-free asset with price $A(t)$ and one risky asset with price $S(t)$ at time $t$ satisfying the following stochastic differential equations.

\begin{align*}
  \mathrm{d}A(t) &= r A(t) \mathrm{d}t,\\
  \mathrm{d}S(t) &= \mu S(t) \mathrm{d}t + \sigma S(t) \mathrm{d}W(t),
\end{align*}

where $r$ is the risk-free rate, $W$ is a Wiener process with respect to the physical probability, $\mu$ is the drift and $\sigma$ the volatility.

In this section, we expand this model in two ways to arrive at the multi-asset Black-Scholes model with variable coefficients. First, we extend to $d$ risky assets $S_j(t), j =  1, \ldots, d$, each driven by $d$ independent Wiener process $W_i(t), i = 1, \ldots, d$. Second, the coefficients are now functions of time. The dynamics of risky assets becomes

\begin{align}
  \mathrm{d}S_i(t) = \mu_i(t) S_i(t) \mathrm{d}t + \sum_{j=1}^{d} c_{ij}(t) S_i(t) \mathrm{d}W_j(t),
\end{align}

for $i=1,\ldots,d$. Some further assumptions are required on the coefficients: $\mu_i(t), c_{ij}(t)$ are adapted to the filtration generated by the Wiener processes, have continuous paths and are bounded by a deterministic constant.

FIXME: Justificiation of $d$ Wiener processes for $d$ assets?

FIXME: Comment about the fact that assumptions are justified in the following development + references to where.

FIXME: Solution to Black-Scholes equation

FIXME: Recap of what will be done

FIXME: Construction of a risk-neutral probability

FIXME: Definition of contigent claim

FIXME: Definition of strategy

FIXME: Self-financing condition

FIXME: Definition of martingale strategy

FIXME: Definition of admissible strategy

FIXME: Explanation as to why we need admissiblity?

FIXME: Definition of replicating strategy

FIXME: Representation of contigent claims as processes assumed to be It\^o

FIXME: Completeness of the model

FIXME: Pricing with risk-neutral expectations + comment about unknown joint distribution

FIXME: Black-Scholes PDE

\section{The Fokker-Planck equation}

% Introduce the Fokker-Planck equation. Use [4] as a reference.

FIXME

\section{Dupire's equation}

% Give a detailed derivation of the Dupire’s equation (equation starting with ∂C = on page 171 in [2]). ∂T
% Use section 2 from [2] and the section ‘The continuous time theory’ from [3] as a source for the proof.

FIXME

\section{Generalisation to multiple assets}

% Provide the setup and give a detailed proof of Theorem 1 from [2]. This should be based on section 3 from [2]

FIXME

\section{Alternative proof}

% Present the alternative proof of Theorem 1, anded on Appendix A from [2].

FIXME

\section{Numerical example}

% Give a numerical example of how Theorem 1 can be applied to recover aij. You can restrict to the simplest setting of a two dimensional Black-Scholes model.

FIXME

\section{References}

FIXME

\end{document}
